\documentclass[a4paper]{report}

\usepackage[utf8]{inputenc}
\usepackage[T1]{fontenc}
\usepackage[french]{babel}
\usepackage{amsmath}
\usepackage{graphicx}
\usepackage{numprint}
\usepackage{enumitem}
\usepackage{scrextend}
\usepackage{a4wide}
\usepackage{float}
\usepackage{textcomp, gensymb}
\usepackage{booktabs}
\usepackage{array}
\usepackage{tabularx}
\usepackage{minted}
\usepackage{algpseudocode}
\usepackage{rotating}
\usepackage[toc,page]{appendix}
\usepackage[final]{pdfpages}
\usepackage[footnote]{acronym}
\usepackage[export]{adjustbox}
\usepackage[lofdepth,lotdepth]{subfig}
\usepackage{hyperref}
\usepackage[colorinlistoftodos]{todonotes}

\renewcommand{\labelenumii}{\theenumii}
\renewcommand{\theenumii}{\theenumi.\arabic{enumii}.}

\let\EndItemize\enditemize
\def\enditemize{\EndItemize\bigskip}

\newenvironment{absolutelynopagebreak}
  {\par\nobreak\vfil\penalty0\vfilneg
   \vtop\bgroup}
  {\par\xdef\tpd{\the\prevdepth}\egroup
   \prevdepth=\tpd}

\setlength{\parindent}{2em}
\setlength{\parskip}{1em}
\addtokomafont{labelinglabel}{\bf}

\begin{document}

%%%%%%%%%%%%%%%%%%
%%% First page %%%
%%%%%%%%%%%%%%%%%%

\begin{titlepage}
\begin{center}

\includegraphics[width=0.6\textwidth]{img/logo_polytech.png}\\[1cm]

{\large Informatique Industrielle en apprentissage}\\[0.5cm]

{\large Projet de Fin d'Étude}\\[0.5cm]

% Title
\rule{\linewidth}{0.5mm} \\[0.4cm]
{ \huge \bfseries Or-Box - Rapport final\\[0.4cm] }
\rule{\linewidth}{0.5mm} \\[1.5cm]

% Author and supervisor
\noindent
\begin{minipage}{0.4\textwidth}
  \begin{flushleft} \large
    \emph{Auteur :}\\
    M.~Pierre \textsc{Robillard}\\
  \end{flushleft}
\end{minipage}%
\begin{minipage}{0.4\textwidth}
  \begin{flushright} \large
    \emph{Encadrant :} \\
    M.~Frédéric \textsc{Rayar}\\
  \end{flushright}
\end{minipage}

\vfill

% Bottom of the page
{\large Version 0.1 du\\ \today}

\end{center}
\end{titlepage}

\tableofcontents
\listoffigures

\chapter{Introduction}
    \section{Domaine d'application}
    
    Le domaine d’application se limite à la OrBox --- un PFE pour le département d'Informatique Industrielle de Polytech Tours.
    Pour rappel, la Or-Box est un système destiné aux personnes souffrant de déficience visuelle.
    Elle a pour but de les aider au quotidien à reconnaître et distinguer des objets de la vie courante, et ce malgré leur handicap.
    
    \section{Porté du document}
    
    L’objet du document est de faire un retour d'expérience sur le déroulement du PFE dans son ensemble.
    Il se concentrera notablement sur la gestion de projet et les résultats obtenues.
    
    Pour mémoire, les spécifications du système sont décrites dans \cite{OBCdS}, la modélisation et l'analyse du système sont décrites dans \cite{OBMod}. La lecture préalable de ces derniers est recommandée.
    
\chapter{Retour d'expérience sur la gestion de projet}
\section{Planning prévisionnel \& réel}

La figure \ref{ganttinit} reprend le planning prévisionnel présenté dans le cahier de spécification, seulement elle n'en garde que les macros-tâches par souci de lisibilités.
La figure \ref{ganttfinal} montre à la même échelle l'avancement réel sur ces macros-tâches.
Les écarts notables sont :
\begin{description}
    \item[Tâche "électronique"] La fin réelle a été mi-décembre, au lieu de la planification début novembre.
    Ce retard est du a trois facteurs ; le premier concerne une légère sous-estimation du temps d'assemblage qui compte pour environ une semaine de délai, le deuxième concerne un problème technique lors de la phase de test (2 semaines) et le troisième concerne un problème de ressource pour la fabrication du PCB (3 semaines) --- plus de détail sur ces deux derniers points dans la section \ref{diff}.
    \item[Tâche "Image processing \& ML"] L'intégration de l'électronique dans le boitier étant une précédence à l'acquisition des images formant la base d'apprentissage, le travail sur le traitement d'image a été repoussé à mi-décembre.
    \item[Tâche "Serveur HTTP"] Cette tâche n'était pas prioritaire et ne faisait pas partie du chemin,  sa réalisation a été repoussée en vue de se concentrer sur la rédaction du cahier de spécification sur la période temps qui lui était initialement alloué.
\end{description}

\begin{figure}[H]
    \centering
    \includegraphics[width=\textwidth]{ganttproInit}
    \caption{Planning prévisionnel}
    \label{ganttinit}
\end{figure}

\begin{figure}[H]
    \centering
    \includegraphics[width=\textwidth]{ganttproFinal}
    \caption{Planning réel}
    \label{ganttfinal}
\end{figure}

\section{Difficultés rencontrées}
\label{diff}


\chapter{Avancement du projet}
\section{Écart avec la spécification}
Voici la liste des différences entre la spécification et l'implémentation :
\begin{description}
    \item[L'interface d'administration.] Il s'agit de différences technologiques.
    D'abord envisagé comme étant une application écrite en Scala avec le framework Scalatra et une sérialisation des données sur MongoDB afin de monter en compétences sur ces nouvelles technologies, elle a finalement été écrite avec NodeJS et la sérialisation des données faites par écriture de fichiers JSON.
    Ce changement de technologies est dû à un manque de temps pour l'autoformation.
    \item[L'interface d'IHM embarqué] en charge du lancement d'un scénario de reconnaissance sur l'appuie du bouton physique de la Orbox n'a pas été écrite.
    Afin de tester toutefois ce scénario crucial, un bouton logiciel sur l'interface d'administration a été ajouté.
\end{description}

\section{Résultats des tests de performance}

\begin{table}[H]
\centering
\resizebox{\textwidth}{!}{%
\begin{tabular}{@{}llllllllllll@{}}
\toprule
\multicolumn{2}{l}{\textbf{Classe}} & \multicolumn{2}{l}{\textbf{Perimetère {[}px{]}}} & \multicolumn{2}{l}{\textbf{Circularité {[}\%{]}}} & \multicolumn{2}{l}{\textbf{Aire  {[}px^{2}{]}}} & \multicolumn{2}{l}{\textbf{\begin{tabular}[c]{@{}l@{}}Hauteur du rectangle\\ circonscrit {[}px{]}\end{tabular}}} & \multicolumn{2}{l}{\textbf{\begin{tabular}[c]{@{}l@{}}Largeur du rectangle\\ circonscrit {[}px{]}\end{tabular}}} \\
\textbf{Nom} & \textbf{Population} & \textbf{$\bar{x}$} & \textbf{$\sigma$} & \textbf{$\bar{x}$} & \textbf{$\sigma$} & \textbf{$\bar{x}$} & \textbf{$\sigma$} & \textbf{$\bar{x}$} & \textbf{$\sigma$} & \textbf{$\bar{x}$} & \textbf{$\sigma$} \\ \midrule
1 cent tail & 146 & 459.773 & \textit{24.206} & 0.891 & \textit{0.043} & 14,986.616 & \textit{1,315.304} & 126.607 & \textit{6.984} & 141.106 & \textit{9.116} \\
2 cents tail & 185 & 538.197 & \textit{28.429} & 0.892 & \textit{0.036} & 20,577.589 & \textit{1,955.259} & 151.506 & \textit{9.225} & 163.784 & \textit{10.186} \\
5 cents tail & 100 & 600.623 & \textit{24.398} & 0.886 & \textit{0.05} & 25,470.67 & \textit{2,399.964} & 169.136 & \textit{8.064} & 183.457 & \textit{9.643} \\
10 cents tail & 91 & 554.009 & \textit{17.996} & 0.892 & \textit{0.041} & 21,845.357 & \textit{1,931.995} & 156.86 & \textit{9.101} & 168.87 & \textit{8.329} \\
20 cents tail & 131 & 630.765 & \textit{20.307} & 0.884 & \textit{0.05} & 27,977.782 & \textit{2,009.784} & 177.785 & \textit{8.471} & 191.908 & \textit{8.294} \\
50 cents tail & 95 & 686.518 & \textit{24.688} & 0.881 & \textit{0.064} & 32,990.795 & \textit{2,636.762} & 194.756 & \textit{8.917} & 207.497 & \textit{11.377} \\
1 euro tail & 107 & 665.33 & \textit{29.773} & 0.874 & \textit{0.082} & 30,641.626 & \textit{2,144.526} & 187.928 & \textit{5.305} & 200.429 & \textit{8.841} \\
2 euros tail & 66 & 717.925 & \textit{8.798} & 0.903 & \textit{0.012} & 37,058.674 & \textit{1,017.615} & 206.697 & \textit{5.75} & 218.561 & \textit{4.034} \\ \bottomrule
\end{tabular}%
}
\caption{Moyenne et écart par classe pour les descripteurs basé sur les contours}
\label{stats}
\end{table}

Avec le retard pris pour la fabrication du circuit imprimé, il m'a resté moins de temps qu'escompté pour testé toutes les architectures architectures de classificateurs imaginés.
Heureusement la qualité des descripteurs choisis a permis d'atteindre des résultats satisfaisants rapidement.

Afin de mesurer la qualité des descripteurs la moyenne et l'écart type de chaque a été calculés.
Le résultat des ces mesures est visible dans la table \ref{stats}, le même travail a été effectué sur les histogrammes de teintes.
Ces calculs et leurs représentations graphiques ont été intégrés à l'interface d'administration du système afin d'aider l'utilisateur à paramétrer correctement sa Orbox.
On arrive ainsi à confirmer notre prédiction que le descripteur \emph{circularité} ne discrime pas suffisament les différentes classes entre elles dans le cas des pièces de monnaie.

\begin{table}[H]
\centering
\resizebox{\textwidth}{!}{%
\begin{tabular}{@{}lll@{}}
\toprule
\textbf{Descripteurs utilisés} & \textbf{Classificateur} & \textbf{Erreur moyenne} \\ \midrule
Aire et 60 barres d'histogramme & 1-NN 150-fold & 20.17\% \\ \midrule
Aire et 60 barres d'histogramme & 9-NN 150-fold & 22.14 \% \\ \hline
\begin{tabular}[c]{@{}l@{}}Aire, Perimètre, hauteur et largeur du rectangle\\ 15/42 barres d'histogramme\end{tabular} & SVM RBF 20-fold & 12.61\% \\ \midrule
\begin{tabular}[c]{@{}l@{}}Aire, Perimètre, hauteur et largeur du rectangle\\ 15/42 barres d'histogramme\end{tabular} & SVM Sigmoïd 20-fold & 78.23\% \\ \midrule
\begin{tabular}[c]{@{}l@{}}Aire, Perimètre, hauteur et largeur du rectangle\\ 15/42 barres d'histogramme\end{tabular} & SVM CHI2 20-fold & 7.42\% \\
\bottomrule
\end{tabular}%
}
\caption{Résultat obtenu pour 921 échantillons parmi 8 classes}
\label{SVM}
\end{table}

\chapter{Rapport financier}

Les coûts financiers de la Orbox ont été répartis sur les deux années.
Le tableau \ref{finance} donne l'estimation du prix du prototype.
Il est difficile d'estimer le coûts de production à partir de celui-ci.
En effet de nombreuses économies d'échelle seraient possibles ; aussi il souvent imposé une quantité minimum sur l'achat des composants électroniques.
L'annexe \ref{farnell} donne le détail de cette commande, on y voit que de nombreuses références n'apparaissant qu'une fois sur la BOM --- i.e. des résistances --- ont nécessité l'achat de 10 composants discrets.
Cependant tous n'ont pas été achetés en vain, car n'étant pas soudeur-câbleur de formation mon taux de soudure raté est élevé.

\begin{table}[H]
\centering
\begin{tabular}{@{}lll@{}}
\toprule
\textbf{Commandes} & \textbf{Fournisseurs} & \textbf{Coût} \\ \midrule
Composants électronique & Farnell & 128.75 € \\ 
Fabrication PCB & Euro-cricuit & 37.84 € \\ 
Camera USB & Amazon & 47.40 € \\ 
Raspberry Pi & RadioSpare & 34.95 € \\ 
Impression 3D & Polytech & - \\ 
Plexyglass, diffuseur   & & \textit{Don} \\ \midrule
\multicolumn{2}{r}{\textbf{Total}} & \textit{248.94 €} \\ \bottomrule
\end{tabular}
\caption{Récapitulatif des achats}
\label{finance}
\end{table}

\chapter{Conclusion}
Malgré une gestion du temps plus dure que prévu, le projet aboutit finalement sur une réussite des objectifs convenus avec la MOA en début de projets.
C'est à dire, qu'un prototype fonctionnel a été obtenu --- originalement, il avait été stipulé \emph{pleinement fonctionnel}, ces dernières finitions seront apportés dans le cadre des portes ouvertes --- l'objectif souhaité d'un taux de reconnaissance de 90\% minimum le cas d'utilisation des pièces de monnaie d'euros aété atteint.

\bibliographystyle{alpha}
\bibliography{mybib}
\begin{appendices}
\chapter{API des Web Services}
\label{annexe_api}

\section{Objects}


\begin{absolutelynopagebreak}
\paragraph{Lister les objets}

Permet d'obtenir la liste des ressources \emph{objects} pouvant servir à l'apprentissage.

\begin{tabular}{@{}p{2cm}p{11.5cm}@{}}
    \toprule
    \textbf{Verbe}                        & GET \hspace{2.5cm} \textbf{URI} \hspace{0.25cm} /api/objects   \\ \midrule
    \midrule
    \textbf{Réponses}                     &        \\
    \multicolumn{1}{r}{\textit{Code}}   & 200 OK \\
    \multicolumn{1}{r}{\textit{Body}}   & \begin{minted}{JSON}
{
  "_links": {
    "self": {"href": "http://orbox.io/api/objects"},
    "objects": [
      {"href": "http://orbox.io/api/objects/1a2b3c"},
      {"href": "http://orbox.io/api/objects/d4e5f6"}
    ]
  }
}
    \end{minted}
    \\ \bottomrule
\end{tabular}
\end{absolutelynopagebreak}
 
\begin{absolutelynopagebreak}
\paragraph{Obtenir un objet}

Permet d'obtenir les informations relatives à une classe d'objets.

\begin{tabular}{@{}p{2cm}p{11.5cm}@{}}
    \toprule
    \textbf{Verbe}                        & GET \hspace{2.5cm} \textbf{URI} \hspace{0.25cm} /api/objects/:id   \\ \midrule
    \midrule
    \textbf{Réponses}                     &        \\
    \multicolumn{1}{r}{\textit{Code}}   & 200 OK \\
    \multicolumn{1}{r}{\textit{Body}}   & \begin{minted}{JSON}
{
  "_links": {
    "self": {"href": "http://orbox.io/api/objects/1a2b3c"},
    "rois": [
      {"href": "http://orbox.io/api/rois/12az3e4r"},
      {"href": "http://orbox.io/api/rois/qs78d9e1"}
    ]
  },
  "name": "Pièce 2€ face FR",
  "description": "Coté face (national - FR) des pièces de 2€",
  "audio_fr": "pièce de 2€",
  "is_male_fr": true,
  "audio_en": "2€ coin"
}
    \end{minted}
    \label{jsonHalObjects}
    \\ \bottomrule
\end{tabular}
\end{absolutelynopagebreak}

\begin{absolutelynopagebreak}
\paragraph{Créer un objet}

Si un objet a deux ou plusieurs faces bien distinctes alors plusieurs ressources \emph{objects}  devront être crée avec une même description audio associé --- par exemple, dans le cas d'une pièce d'euro on doit être créer une ressource \emph{objects} pour la face internationale et une ressource \emph{objects} pour chaque face nationale.

\begin{tabular}{@{}p{2cm}p{11.5cm}@{}}
    \toprule
    \textbf{Verbe}                        & POST \hspace{2.5cm} \textbf{URI} \hspace{0.25cm} /api/objects   \\ \midrule
    \textbf{Paramètres}                   &        \\
    \multicolumn{1}{r}{\textit{name}} & [string] nom de la classe  \\ 
    \multicolumn{1}{r}{\textit{description}} & [string] (optionnelle) description supplémentaire sur l'objet  \\ 
    \multicolumn{1}{r}{\textit{audio\_fr}} & [string]  (optionnelle) message audio délivré quand le profile est en français \\ 
    \multicolumn{1}{r}{\textit{is\_male\_fr}} & [boolean] (optionnelle) permet d'accorder le pronom déterminé du singulier (un ou une) lors du message audio --- par défaut le masculin sera utilisé.   \\
    \multicolumn{1}{r}{\textit{audio\_en}} & [string] (optionnelle) message audio délivré quand le profile est en anglais~---~par défaut le texte français sera lu.  \\
    \multicolumn{1}{r}{\textit{rois}} & [array of string] (optionnelle) Passer en paramètre la liste des id des ressources \emph{ROIs} associés à l'objet. \\ \midrule
    \textbf{Réponses}                     &        \\
    \multicolumn{1}{r}{\textit{Code}}   & 201 Created \\
    \multicolumn{1}{r}{\textit{Location}}   & http://orbox.io/api/objects/1a2b3c \\\multicolumn{1}{r}{\textit{Body}}   & Voir \emph{Body} de la méthode GET /api/objects/:id pour le format du JSON retourné (p.\pageref{jsonHalObjects}). \\ \bottomrule
\end{tabular}
\end{absolutelynopagebreak}

\begin{absolutelynopagebreak}
\paragraph{Mettre à jour un objet}

Permet de mettre à jour les données d'un \emph{objects}, notamment d'associer des \emph{ROIs}.

\begin{tabular}{@{}p{2cm}p{11.5cm}@{}}
    \toprule
    \textbf{Verbe}                        & PUT \hspace{2.5cm} \textbf{URI} \hspace{0.25cm} /api/objects/:id   \\ \midrule
    \textbf{Paramètres}                   &        \\
    \multicolumn{1}{r}{\textit{name}} & [string] nom de la classe  \\ 
    \multicolumn{1}{r}{\textit{description}} & [string] description supplémentaire sur l'objet  \\ 
    \multicolumn{1}{r}{\textit{audio\_fr}} & [string] message audio délivré quand le profile est en français \\ 
    \multicolumn{1}{r}{\textit{is\_male\_fr}} & [boolean] permet d'accorder en genre le pronom indéterminé au singulier lors du message audio.   \\
    \multicolumn{1}{r}{\textit{audio\_en}} & [string] message audio délivré quand le profile est en anglais.  \\
    \multicolumn{1}{r}{\textit{rois}} & [array of string] Passer en paramètre un tableau vide pour effacer toutes les ressources \emph{ROIs} associées à l'objet, sinon envoyer la liste des id associés. \\ \midrule
    \textbf{Réponses}                     &        \\
    \multicolumn{1}{r}{\textit{Code}}   & 200 OK \\
    \multicolumn{1}{r}{\textit{Body}}   & Voir \emph{Body} de la méthode GET /api/objects/:id pour le format du JSON retourné (p.\pageref{jsonHalObjects}).
    \\ \bottomrule
\end{tabular}
\end{absolutelynopagebreak}

\begin{absolutelynopagebreak}
\paragraph{Supprimer un objet}

Supprime la ressource objet, les \emph{ROIs} associées à cet objet ne seront pas supprimés.

\begin{tabular}{@{}p{2cm}p{11.5cm}@{}}
    \toprule
    \textbf{Verbe}                        & DELETE \hspace{2.5cm} \textbf{URI} \hspace{0.25cm} /api/objects/:id   \\ \midrule
    \textbf{Réponses}                     &        \\
    \multicolumn{1}{r}{\textit{Code}}   & 204 No Content \\ \bottomrule
\end{tabular}
\end{absolutelynopagebreak}

%section apiobjects (end)

\begin{absolutelynopagebreak}
\section{Screenshots}

\paragraph{Lister les screenshots}

Permet d'obtenir la liste des ressources \emph{screenshots}.

\begin{tabular}{@{}p{2cm}p{11.5cm}@{}}
    \toprule
    \textbf{Verbe}                        & GET \hspace{2.5cm} \textbf{URI} \hspace{0.25cm} /api/screenshots   \\ \midrule
    \textbf{Paramètres}                   &        \\ \midrule
    \textbf{Réponses}                     &        \\
    \multicolumn{1}{r}{\textit{Code}}   & 200 OK \\
    \multicolumn{1}{r}{\textit{Body}}   & \begin{minted}{JSON}
{
  "_links": {
    "self": {"href": "http://orbox.io/api/screenshots"},
    "screenshots": [
      {"href": "http://orbox.io/api/screenshots/1a2b3c"},
      {"href": "http://orbox.io/api/screenshots/d4e5f6"}
    ]
  }
}
\end{minted}
    \\ \bottomrule
\end{tabular}
\end{absolutelynopagebreak}
  
\begin{absolutelynopagebreak}
\paragraph{Obtenir un screenshot}

Permet d'obtenir les informations relatives à une ressource \emph{screenshots}.
Ces informations sont :
\begin{labeling}[~--]{thumbnail}
    \item [raw\_lit] lien vers l'image brute (sans traitement) capturé avec l'éclairage.
    \item [raw\_dark] lien vers l'image brute (sans traitement) capturé sans éclairage.
    \item [und\_lit] lien vers l'image réaplani et filtré résultant des étapes de pré-processing, capturé avec l'éclairage.
    \item [und\_dark] lien vers l'image réaplani et filtré résultant des étapes de pré-processing, capturé sans l'éclairage. 
    \item [thumbnail] lien vers la miniature de l'image \emph{und\_lit}.
    \item [timestamp] horodatage des images capturées.
    \item [rois] la liste des régions d'intérêts trouvés dans le screenshot, c'est-à-dire le résultat de l'étape de segmentation.
\end{labeling}


\begin{tabular}{@{}p{2cm}p{11.5cm}@{}}
    \toprule
    \textbf{Verbe}                        & GET \hspace{2.5cm} \textbf{URI} \hspace{0.25cm} /api/screenshots/:id   \\ \midrule
    \textbf{Réponses}                     &        \\
    \multicolumn{1}{r}{\textit{Code}}   & 200 OK \\
    \multicolumn{1}{r}{\textit{Body}}   & \begin{minted}{JSON}
{
  "_links": {
    "self": {"href": "http://orbox.io/api/screenshots/1a2b3c"},
    "raw_lit": {"href": "http://orbox.io/api/img/14qa7d"},
    "raw_dark": {"href": "http://orbox.io/api/img/89de5f"},
    "und_lit": {"href": "http://orbox.io/api/img/47de9f"},
    "und_dark": {"href": "http://orbox.io/api/img/68dea8"},
    "thumbnail": {"href": "http://orbox.io/api/img/48dgr3"},
    "rois" : [
      {"href": "http://orbox.io/api/rois/4d7gt5"},
      {"href": "http://orbox.io/api/rois/64adcs"},
      {"href": "http://orbox.io/api/rois/fr8qfr"}
    ]
  },
  "timestamp": "2016-04-23T18:25:43.511Z"
}
    \end{minted}
    \label{jsonHalScreenshots}
    \\ \bottomrule
\end{tabular}
\end{absolutelynopagebreak}

\begin{absolutelynopagebreak}
\paragraph{Créer un screenshots}
Cette action fait appel à la caméra de la OrBox ; deux photos sont prises, les étapes de pré-processing et de segmentation sont effectuées.

\begin{tabular}{@{}p{2cm}p{11.5cm}@{}}
    \toprule
    \textbf{Verbe}                        & POST \hspace{2.5cm} \textbf{URI} \hspace{0.25cm} /api/screenshots   \\ \midrule
    \textbf{Réponses}                     &        \\
    \multicolumn{1}{r}{\textit{Code}}   & 201 Created \\
    \multicolumn{1}{r}{\textit{Location}}   & http://orbox.io/api/screenshots/1a2b3c \\\multicolumn{1}{r}{\textit{Body}}   & Voir \emph{Body} de la méthode GET /api/screenshots/:id pour le format du JSON retourné (p.\pageref{jsonHalScreenshots}). \\ \bottomrule
\end{tabular}
\end{absolutelynopagebreak}

\begin{absolutelynopagebreak}
\paragraph{Mettre à jour un screenshot}

Permet de mettre à jour les donnés d'un \emph{screenshots}.
Cette action est à utiliser en cas de changement d'algorithme de segmentation ou de pré-processing.

\begin{tabular}{@{}p{2cm}p{11.5cm}@{}}
    \toprule
    \textbf{Verbe}                        & PUT \hspace{2.5cm} \textbf{URI} \hspace{0.25cm} /api/screenshots/:id   \\ \midrule
    \textbf{Paramètres}                   &        \\
    \multicolumn{1}{r}{\textit{und\_lit}} & [string] id de l'image réaplani et filtré résultant des étapes de pré-processing, capturée avec l'éclairage  \\ 
    \multicolumn{1}{r}{\textit{und\_dark}} & [string] id de l'image réaplani et filtré résultant des étapes de pré-processing, capturée sans l'éclairage  \\ 
    \multicolumn{1}{r}{\textit{thumbnail}} & [string] id de la miniature de la nouvelle image \emph{und\_lit} \\ 
    \multicolumn{1}{r}{\textit{rois}} & [array of string] Passer en paramètre un tableau vide pour effacer toutes les ressources \emph{ROIs} associées au screenshots, sinon envoyer la liste des id associés. \\ \midrule
    \textbf{Réponses}                     &        \\
    \multicolumn{1}{r}{\textit{Code}}   & 200 OK \\
    \multicolumn{1}{r}{\textit{Body}}   & Voir \emph{Body} de la méthode GET /api/screenshots/:id pour le format du JSON retourné (p.\pageref{jsonHalScreenshots}).
    \\ \bottomrule
\end{tabular}
\end{absolutelynopagebreak}

\begin{absolutelynopagebreak}
\paragraph{Supprimer un screenshots}
Toutes les images associées seront aussi supprimées, mais pas les \emph{ROIs}.

\begin{tabular}{@{}p{2cm}p{11.5cm}@{}}
    \toprule
    \textbf{Verbe}                        & DELETE \hspace{2.5cm} \textbf{URI} \hspace{0.25cm} /api/objects/:id   \\ \midrule
    \textbf{Réponses}                     &        \\
    \multicolumn{1}{r}{\textit{Code}}   & 204 No Content \\ \bottomrule
\end{tabular}
\end{absolutelynopagebreak}
 
%section apiscreenshots (end)

\begin{absolutelynopagebreak}
\section{ROIs}

\paragraph{Lister les régions d'intéréts}

Permet d'obtenir la liste des ressources \emph{ROIs}.

\begin{tabular}{@{}p{2cm}p{11.5cm}@{}}
    \toprule
    \textbf{Verbe}                        & GET \hspace{2.5cm} \textbf{URI} \hspace{0.25cm} /api/rois   \\ \midrule
    \textbf{Réponses}                     &        \\
    \multicolumn{1}{r}{\textit{Code}}   & 200 OK \\
    \multicolumn{1}{r}{\textit{Body}}   & \begin{minted}{JSON}
{
  "_links": {
    "self": {"href": "http://orbox.io/api/rois"},
    "rois": [
      {"href": "http://orbox.io/api/rois/1a2b3c"},
      {"href": "http://orbox.io/api/rois/d4e5f6"}
    ]
  }
}
    \end{minted}
    \\ \bottomrule
\end{tabular}
\end{absolutelynopagebreak}
  
\begin{absolutelynopagebreak}
\paragraph{Obtenir une région d'intérêt}

Permet d'obtenir les informations relatives à une ressource \emph{rois}.
Il s'agit d'une image et de ses métadonnées résultant de l'opération de segmentation.
Ces informations sont :
\begin{labeling}[~--]{top\_left\_x, top\_left\_y}
    \item [parent] lien vers le \emph{screenshots} à partir duquel la région d'intérêt a été calculée.
    \item [img] lien vers l'image --- cette image est réalisée à partir de l'image \emph{und\_lit} de parent.
    \item [width, height] largeur et hauteur de la région d'intérêt.
    \item [top\_left\_x, top\_left\_y] coordonné du coin supérieur gauche de la région d'intérêt dans l'image du \emph{parent}.
\end{labeling}

\begin{tabular}{@{}p{2cm}p{11.5cm}@{}}
    \toprule
    \textbf{Verbe}                        & GET \hspace{2.5cm} \textbf{URI} \hspace{0.25cm} /api/rois/:id   \\ \midrule
    \textbf{Réponses}                     &        \\
    \multicolumn{1}{r}{\textit{Code}}   & 200 OK \\
    \multicolumn{1}{r}{\textit{Body}}   & \begin{minted}{JSON}
{
  "_links": {
    "self": {"href": "http://orbox.io/api/rois/1a2b3c"},
    "img": {"href": "http://orbox.io/api/img/48dgr3"},
    "parent": {"href": "http://orbox.io/api/screenshots/89de5f"}
  },
  "top_left_x": 249,
  "top_left_y": 725,
  "width": 64,
  "height": 58,
  "descriptors": [
    { "ORB": OpenCV_Mat_parsedInJSON }, 
    { "BRISK": OpenCV_Mat_parsedInJSON }
  ]
}
    \end{minted}
    \label{jsonHalROIs}
    \\ \bottomrule
\end{tabular}
\end{absolutelynopagebreak}

\begin{absolutelynopagebreak}
\paragraph{Créer un ROI}
Créer une ressource \emph{ROIs} --- cette action n'est censée être appelée que par l'algorithme de segmentation.

\begin{tabular}{@{}p{2cm}p{11.5cm}@{}}
    \toprule
    \textbf{Verbe}                        & POST \hspace{2.5cm} \textbf{URI} \hspace{0.25cm} /api/rois   \\ \midrule
    \textbf{Paramètres}                   &        \\
    \multicolumn{1}{r}{\textit{parent}}  & [string] id de la ressource \emph{screenshots} à partir duquel la région d'intérêt a été calculé. \\ 
    \multicolumn{1}{r}{\textit{img}}  & [string] id de la ressource \emph{image} associée. \\
    \multicolumn{1}{r}{\textit{topLeftX}}  & [integer] coordonnée horizontale du coin supérieur gauche.  \\ 
    \multicolumn{1}{r}{\textit{topLeftY}}  & [integer] coordonnée verticale du coin supérieur gauche. \\ 
    \multicolumn{1}{r}{\textit{width}}  & [integer] largeur de la région d'intérêt.\\ 
    \multicolumn{1}{r}{\textit{height}}  & [integer] hauteur de la région d'intérêt. \\
    \multicolumn{1}{r}{\textit{descriptors}} & [array of string] (optionnelle) 
    Passer le nom des descripteurs à calculer.
    La liste des noms valides sera déterminée plus tard lors du développement. \\
    \midrule
    \textbf{Réponses}                     &        \\
    \multicolumn{1}{r}{\textit{Code}}   & 201 Created \\
    \multicolumn{1}{r}{\textit{Location}}   & http://orbox.io/api/rois/1a2b3c \\\multicolumn{1}{r}{\textit{Body}}   & Voir \emph{Body} de la méthode GET /api/rois/:id pour le format du JSON retourné (p.\pageref{jsonHalROIs}). \\ \bottomrule
    \end{tabular}
\end{absolutelynopagebreak}

\begin{absolutelynopagebreak}
\paragraph{Mettre à jour un ROI}

Permet de mettre à jour les donnés d'un \emph{roi}, seuls les descripteurs sont mis à jour.

\begin{tabular}{@{}p{2cm}p{11.5cm}@{}}
    \toprule
    \textbf{Verbe}                        & PUT \hspace{2.5cm} \textbf{URI} \hspace{0.25cm} /api/rois/:id   \\ \midrule
    \textbf{Paramètres}                   &        \\
    \multicolumn{1}{r}{\textit{descriptors}} & [array of string] Passer en paramètre un tableau vide pour effacer tous les descripteurs associés.
    Sinon, passer le nom des descripteurs à calculer.
    La liste des noms valides sera déterminée plus tard lors du développement. \\\midrule
    \textbf{Réponses}                     &        \\
    \multicolumn{1}{r}{\textit{Code}}   & 200 OK \\
    \multicolumn{1}{r}{\textit{Body}}   & Voir \emph{Body} de la méthode GET /api/rois/:id pour le format du JSON retourné (p.\pageref{jsonHalROIs}).
    \\ \bottomrule
\end{tabular}
\end{absolutelynopagebreak}

\begin{absolutelynopagebreak}
\paragraph{Supprimer un ROI}
~

\begin{tabular}{@{}p{2cm}p{11.5cm}@{}}
    \toprule
    \textbf{Verbe}                        & DELETE \hspace{2.5cm} \textbf{URI} \hspace{0.25cm} /api/rois/:id   \\ \midrule
    \textbf{Réponses}                     &        \\
    \multicolumn{1}{r}{\textit{Code}}   & 204 No Content \\ \bottomrule
\end{tabular}
\end{absolutelynopagebreak}

% section apiROIs (end)

\begin{absolutelynopagebreak}
\section{Profiles}

\paragraph{Lister les profils}

Permet d'obtenir la liste des ressources \emph{profiles}.

\begin{tabular}{@{}p{2cm}p{11.5cm}@{}}
    \toprule
    \textbf{Verbe}                        & GET \hspace{2.5cm} \textbf{URI} \hspace{0.25cm} /api/profiles   \\ \midrule
    \textbf{Paramètres}                   &        \\
    \multicolumn{1}{r}{\textit{actived}} & [boolean] si "true" répond par un 303, si "false" répond par un 200.  \\
    \midrule
    \textbf{Réponses}                     &        \\
    \multicolumn{1}{r}{\textit{Code}}   & 200 OK \\
    \multicolumn{1}{r}{\textit{Body}}   & \begin{minted}{JSON}
{
  "_links": {
    "self": {"href": "http://orbox.io/api/profiles"},
    "profiles": [
      {"href": "http://orbox.io/api/profiles/1a2b3c"},
      {"href": "http://orbox.io/api/profiles/d4e5f6"}
    ]
  }
}
    \end{minted}
    \\ \midrule
    \multicolumn{1}{r}{\textit{Code}}   & 303 See Other \\
    \multicolumn{1}{r}{\textit{Location}}   & http://orbox.io/api/profiles/1a2b3c \\
    \bottomrule
\end{tabular}
\end{absolutelynopagebreak}
  
\begin{absolutelynopagebreak}
\paragraph{Obtenir un profil}

Permet d'obtenir les informations relatives à une ressource \emph{profiles}.
Il s'agit d'une ressource regroupant les options de configuration de la OrBox.
Les options paramétrables par profil sont :
\begin{labeling}[~--]{classificator\_used}
    \item[active] indique si le profil est le profil utilisé par la Orbox lors des prédictions.
    \item[audio\_langage] La langue à utiliser pour le message audio lors des prédictions.
    \item[descriptor\_used] Descripteur à utiliser. 
    La liste des descripteurs sera détermineée plus tard lors du développement.
    \item[classificator\_used] Algorithme de classification à utiliser.
    La liste des valeurs acceptable n'est pas fixée, mais se limitera probablement à "1NN", "FLANN" et "SVM".
    \item[last\_trained] Horodatage du dernier entrainement.
    \item[last\_updated] Horodatage de la dernière modification faite au training set.
    \item[config\_file] Lien vers le fichier de configuration, résultat de la dernière opération d'entraînement.
    \item[result\_file] Lien vers le fichier des résultats (taux de reconnaissance, nombre d'itérations...) de la dernière opération d'entraînement.
    \item [training\_set] Un ensemble de ressource \emph{objects} sur lesquelles sera entraîné l'algorithme de classification.
    On pourra avoir, par exemple, un profil entrainé sur tous les \emph{objects} relatifs au cas d'utilisation des pièces, un deuxième contenant tous les \emph{objects} relatifs au cas d'utilisation et un troisième regroupant les deux.
\end{labeling}

\begin{tabular}{@{}p{2cm}p{11.5cm}@{}}
    \toprule
    \textbf{Verbe}                        & GET \hspace{2.5cm} \textbf{URI} \hspace{0.25cm} /api/profiles/:id   \\ \midrule
    \textbf{Réponses}                     &        \\
    \multicolumn{1}{r}{\textit{Code}}   & 200 OK \\
    \multicolumn{1}{r}{\textit{Body}}   & \begin{minted}{JSON}
{
  "_links": {
    "self": {"href": "http://orbox.io/api/profiles/48dgr3"},
    "config_file": {"href": "http://orbox.io/api/xml/d4e5f6"},
    "result_file": {"href": "http://orbox.io/api/xml/47de9f"},
    "training_set": [
      {"href": "http://orbox.io/api/objects/1a2b3c"},
      {"href": "http://orbox.io/api/objects/d4e5f6"}
    ]
  },
  "active": true,
  "audio_language": "FR",
  "descriptor_used": "BRISK+COLORHIST64",
  "classificator_used": "SVM",
  "last_trained": "2017-02-21T14:25:43.511Z",
  "last_updated": "2017-04-13T09:45:23.319Z"
}
    \end{minted}
    \label{jsonHalProfiles}
    \\ \bottomrule
\end{tabular}
\end{absolutelynopagebreak}

\begin{absolutelynopagebreak}
\paragraph{Créer un profile}
Créer une ressource \emph{profiles}.

\begin{tabular}{@{}p{2cm}p{11.5cm}@{}}
    \toprule
    \textbf{Verbe}                        & POST \hspace{2.5cm} \textbf{URI} \hspace{0.25cm} /api/rois   \\ \midrule
    \textbf{Paramètres}                   &        \\
    \multicolumn{1}{r}{\textit{active}}  & [boolean] (optionnelle) false par défaut. \\
    \multicolumn{1}{r}{\textit{audio\_language}}  & [string] (optionnelle) Valeurs acceptées "FR" ou "EN".
    Par défaut "FR". \\
    \multicolumn{1}{r}{\textit{descriptor\_used}}  & [string] Nom du descripteur utilisé.
    Les valeurs acceptés ne sont pas encore définies.\\
    \multicolumn{1}{r}{\textit{classificator\_used}}  & [string] Nom du classificateur utilisé.
    Les valeurs acceptées ne sont pas encore définies.\\
    \multicolumn{1}{r}{\textit{training\_set}}  & [array of string] (optionnelles) ids des ressources \emph{objects} constituant le training set.\\
    \midrule
    \textbf{Réponses}                     &        \\
    \multicolumn{1}{r}{\textit{Code}}   & 201 Created \\
    \multicolumn{1}{r}{\textit{Location}}   & http://orbox.io/api/profiles/1a2b3c \\\multicolumn{1}{r}{\textit{Body}}   & Voir \emph{Body} de la méthode GET /api/rois/:id pour le format du JSON retourné (p.\pageref{jsonHalProfiles}). \\ \bottomrule
\end{tabular}
\end{absolutelynopagebreak}

\begin{absolutelynopagebreak}
\paragraph{Mettre à jour un profil}

Permet de mettre à jour les donnés d'un \emph{profiles}.

\begin{tabular}{@{}p{2cm}p{11.5cm}@{}}
    \toprule
    \textbf{Verbe}                        & PUT \hspace{2.5cm} \textbf{URI} \hspace{0.25cm} /api/profiles/:id   \\ \midrule
    \textbf{Paramètres}                   &        \\
    \multicolumn{1}{r}{\textit{active}}  & [boolean] \\
    \multicolumn{1}{r}{\textit{audio\_language}}  & [string] Valeurs acceptées "FR" ou "EN". \\
    \multicolumn{1}{r}{\textit{descriptor\_used}}  & [string] Nom du descripteur utilisé.
    Les valeurs acceptées ne sont pas encore définies.\\
    \multicolumn{1}{r}{\textit{classificator\_used}}  & [string] Nom du classificateur utilisé.
    Les valeurs acceptées ne sont pas encore définies.\\
    \multicolumn{1}{r}{\textit{training\_set}}  & [array of string] Passer en paramètre un tableau vide pour effacer toutes les ressources \emph{objects} constituant le training set du profile, sinon envoyer la liste des id associés. \\
    \midrule
    \textbf{Réponses}                     &        \\
    \multicolumn{1}{r}{\textit{Code}}   & 200 OK \\
    \multicolumn{1}{r}{\textit{Body}}   & Voir \emph{Body} de la méthode GET /api/rois/:id pour le format du JSON retourné (p.\pageref{jsonHalProfiles}).
    \\ \bottomrule
\end{tabular}
\end{absolutelynopagebreak}

\begin{absolutelynopagebreak}
\paragraph{Supprimer un profil}
~

\begin{tabular}{@{}p{2cm}p{11.5cm}@{}}
    \toprule
    \textbf{Verbe}                        & DELETE \hspace{2.5cm} \textbf{URI} \hspace{0.25cm} /api/profiles/:id   \\ \midrule
    \textbf{Réponses}                     &        \\
    \multicolumn{1}{r}{\textit{Code}}   & 204 No Content \\ \bottomrule
\end{tabular}
\end{absolutelynopagebreak}

\begin{absolutelynopagebreak}
\paragraph{Effectuer l'entraînement d'un profil}
\label{ApiProfilesTrain}
Suivant l'algorithme de classification utilisé et la taille du training set, l'entraînement peut prendre beaucoup de temps.
Afin de ne pas bloquer le client, une ressource \emph{queue} est créée; il pourra grâce à elle vérifier périodiquement l'avancement de l'entraînement.
À la fin de l'entraînement les champs "last\_trained", "config\_file" et "result\_file" la ressource \emph{profile} associé seront mis à jour.

\begin{tabular}{@{}p{2cm}p{11.5cm}@{}}
    \toprule
    \textbf{Verbe}                        & GET \hspace{2.5cm} \textbf{URI} \hspace{0.25cm} /api/profiles/:id/train   \\ \midrule
    \textbf{Réponses}                     &        \\
    \multicolumn{1}{r}{\textit{Code}}   & 202 Accepted \\
    \multicolumn{1}{r}{\textit{Location}}   &  http://orbox.io/api/profiles/12ab3c/queue \\
    \bottomrule
\end{tabular}
\end{absolutelynopagebreak}

\begin{absolutelynopagebreak}
\paragraph{Vérifier l'avancement d'un entraînement}
~

\begin{tabular}{@{}p{2cm}p{11.5cm}@{}}
    \toprule
    \textbf{Verbe}                        & GET \hspace{2.5cm} \textbf{URI} \hspace{0.25cm} /api/profiles/:id/queue   \\ \midrule
    \textbf{Réponses}                     &        \\
    \multicolumn{1}{r}{\textit{Code}}   & 200 OK \\
    \multicolumn{1}{r}{\textit{Body}}   & \begin{minted}{JSON}
{
  "_links": {
    "self": {"href": "/api/profiles/d4e5f6/queue"},
    "cancel": {
      "method": "DELETE",
      "href": "/api/profiles/d4e5f6/queue"
    }
  },
  "status": "PENDING"
}
    \end{minted}
    \\ \midrule
    \multicolumn{1}{r}{\textit{Code}}   & 303 See Other \\
    \multicolumn{1}{r}{\textit{Location}}   &  http://orbox.io/api/profiles/12ab3c \\
    \bottomrule
\end{tabular}
\end{absolutelynopagebreak}

\begin{absolutelynopagebreak}
\paragraph{Annuler un entraînement en cours}
~

\begin{tabular}{@{}p{2cm}p{11.5cm}@{}}
    \toprule
    \textbf{Verbe}                        & GET \hspace{2.5cm} \textbf{URI} \hspace{0.25cm} /api/profiles/:id/queue   \\ \midrule
    \textbf{Réponses}                     &        \\
    \multicolumn{1}{r}{\textit{Code}}   & 204 No Content \\
    \bottomrule
\end{tabular}
\end{absolutelynopagebreak}


\begin{absolutelynopagebreak}
\paragraph{Faire une prédiction}
Cette action utilise l'algorithme et le fichier de configuration renseigné dans le profil pour effectuer une prédiction sur l'objet à laquelle appartient chaque région d'intérêts passés en paramètre.

\begin{tabular}{@{}p{2cm}p{11.5cm}@{}}
    \toprule
    \textbf{Verbe}                        & GET \hspace{2.5cm} \textbf{URI} \hspace{0.25cm} /api/profiles/:id/predict   \\ \midrule
    \textbf{Paramètres}                   &        \\
    \multicolumn{1}{r}{\textit{rois}} & [array of string] ids des différentes régions d'intérêts sur lesquelles une prédiction doit être faite.  \\
    \midrule
    \textbf{Réponses}                     &        \\
    \multicolumn{1}{r}{\textit{Code}}   & 200 OK \\
    \multicolumn{1}{r}{\textit{Body}}   & \begin{minted}{JSON}
{
  "_links": {
    "self": {"href": "/api/profiles/d4e5f6/predict"}
  },
  "results": [
    {
      "roi" : {"href": "/api/rois/48dgr3"},
      "prediction":  {"href": "/api/objects/47de9f"}
    },
    {
      "roi" : {"href": "/api/rois/1a2b3c"},
      "prediction":  {"href": "/api/objects/47de9f"}
    }
  ]
}
    \end{minted}
    \\
    \bottomrule
\end{tabular}
\end{absolutelynopagebreak}

%section profiles (end)

\begin{absolutelynopagebreak}
\section{Text to speech}

\paragraph{Obtenir la synthèse vocale d'un texte}
~

\begin{tabular}{@{}p{2cm}p{11.5cm}@{}}
    \toprule
    \textbf{Verbe}                        & GET \hspace{2.5cm} \textbf{URI} \hspace{0.25cm} /api/tts   \\ \midrule
    \textbf{Paramètres}                   &        \\
    \multicolumn{1}{r}{\textit{lang}} & [string] "FR" ou "EN"  \\
    \multicolumn{1}{r}{\textit{text}} & [string] la phrase à synthétisé  \\ \midrule
    \textbf{Réponses}                     &        \\
    \multicolumn{1}{r}{\textit{Code}}   & 200 OK \\
    \multicolumn{1}{r}{\textit{type}}   & audio/wav \\
    \\ \bottomrule
\end{tabular}
\end{absolutelynopagebreak}

\begin{absolutelynopagebreak}
\section{Static}
Un certain nombre de fichiers doivent être manipulés, ils seront stockés soit sur le système de fichier soit sur une base de données --- par exemple \emph{MongoDB} avec \emph{GridFS}.

\begin{tabular}{@{}p{2cm}p{11.5cm}@{}}
    \toprule
    \textbf{Verbe}                        & GET, PUT, DELETE \hspace{2.5cm} \textbf{URI} \hspace{0.25cm} /api/\{img/xml/...\}/:id  \\
    \textbf{Réponses}                     &        \\
    \multicolumn{1}{r}{\textit{Code}}   & 200 OK \\ \bottomrule
\end{tabular}
\end{absolutelynopagebreak}


\end{appendices}

\end{document}
